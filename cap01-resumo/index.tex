\keywords{web scraping; inteligência artificial; modelos de linguagem de grande porte (LLMs); geração aumentada por recuperação (RAG); engenharia de dados; ETL; ciência de dados; bases vetoriais; extração de dados; conformidade; LGPD; governança de dados; automação; ética em dados; inovação tecnológica}
\renewcommand{\abstractname}{Resumo}
\begin{abstract}
Este artigo faz uma revisão sobre o uso de \textit{web scraping} em conjunto com Inteligência Artificial (IA), destacando sua importância para a ciência de dados. O \textit{web scraping} é apresentado dentro do fluxo de Extract--Transform--Load (ETL), com foco na etapa de extração de dados, e também em aplicações mais recentes, como as arquiteturas de \textit{retrieval-augmented generation} (RAG), que usam bases vetoriais para ampliar o conhecimento de modelos de linguagem de grande porte (LLMs). São discutidos ainda aspectos legais e éticos, considerando leis como a Lei Geral de Proteção de Dados (LGPD), além de aplicações práticas em diferentes níveis organizacionais: operacional, tático e estratégico. O trabalho mostra também a evolução do tema, desde as primeiras formas de automação até seu estágio comercial atual. Conclui-se que o \textit{web scraping}, quando aliado à IA, não serve apenas para coletar dados, mas se torna um recurso estratégico para inovação e competitividade, reforçando sua relevância para a área de ciência de dados.
\end{abstract}

\keywords{web scraping; artificial intelligence; large language models (LLMs); retrieval-augmented generation (RAG); data engineering; ETL; data science; vector databases; data extraction; compliance; LGPD; data governance; automation; data ethics; technological innovation}
\renewcommand{\abstractname}{Abstract}
\begin{abstract}
This article reviews the use of web scraping in conjunction with Artificial Intelligence (AI), highlighting its importance for data science. Web scraping is presented within the Extract--Transform--Load (ETL) workflow, focusing on the data extraction step, and also in more recent applications, such as retrieval-augmented generation (RAG) architectures, which use vector databases to augment the knowledge of large language models (LLMs). Legal and ethical aspects are also discussed, considering laws such as the General Data Protection Law (LGPD), as well as practical applications at different organizational levels: operational, tactical, and strategic. The work also shows the evolution of the topic, from the earliest forms of automation to its current commercial stage. It is concluded that web scraping, when combined with AI, not only serves to collect data but becomes a strategic resource for innovation and competitiveness, reinforcing its relevance to the data science field.
\end{abstract}