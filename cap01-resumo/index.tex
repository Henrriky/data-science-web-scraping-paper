\renewcommand{\abstractname}{Resumo}
\begin{abstract}
Este artigo faz uma revisão sobre o uso de \textit{web scraping} em conjunto com Inteligência Artificial (IA), destacando sua importância para a ciência de dados. O \textit{web scraping} é apresentado dentro do fluxo de Extract--Transform--Load (ETL), com foco na etapa de extração de dados, e também em aplicações mais recentes, como as arquiteturas de \textit{retrieval-augmented generation} (RAG), que usam bases vetoriais para ampliar o conhecimento de modelos de linguagem de grande porte (LLMs). São discutidos ainda aspectos legais e éticos, considerando leis como a Lei Geral de Proteção de Dados (LGPD), além de aplicações práticas em diferentes níveis organizacionais: operacional, tático e estratégico. O trabalho mostra também a evolução do tema, desde as primeiras formas de automação até seu estágio comercial atual. Conclui-se que o \textit{web scraping}, quando aliado à IA, não serve apenas para coletar dados, mas se torna um recurso estratégico para inovação e competitividade, reforçando sua relevância para a área de ciência de dados.
\end{abstract}