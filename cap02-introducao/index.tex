\section{Introduction}

O volume de dados disponível cresce de forma acelerada na internet desde o seu “Boom”, impulsionado por redes sociais, comércio eletrônico, artigos, bibliotecas de vídeos e imagens e outras plataformas digitais. Presume-se que, em 2025, o mundo atingirá um marco de 181 zettabytes de dados gerados (Statista, 2025), o que demonstra o desafio na coleta, processamento e transformação dessa quantidade enorme de informações em conhecimento útil para o mundo público ou privado. Dessa forma, torna-se cada vez mais necessário a utilização de técnicas que automatizam a extração dessas informações, sendo capazes de lidar com informações em larga escala.

O Web Scraping é uma técnica que permite a coleta automática de dados da Web, realizando a conversão de páginas Web em informações estruturadas que podem ser analisadas em diferentes contextos de acordo com o propósito final (LOTFI et al., 2022). Quando utilizado em conjunto com e para a Inteligência Artificial (IA), especialmente em modelos de linguagem, permite não apenas a extração, mas também análises avançadas, contextualização e suporte à tomadas de decisão.

O objetivo do artigo é analisar os conceitos de Web Scraping e sua relação com a Inteligência Artificial, destacando suas aplicações práticas, evolução histórica, estágio comercial e perspectivas futuras. Além disso, serão discutidos os aspectos legais e éticos relacionados ao uso dessas técnicas, considerando regulamentações locais como LGPD.

No sentido de atingir esse propósito, o artigo está organizado da seguinte maneira: após essa introdução, apresenta-se uma revisão teórica sobre Web Scraping, Inteligência Artificial e sua integração em pipelines de Extract, Transform, Load (ETL), tal como em arquiteturas Retrieval Augmented Generation (RAG). Em seguida, serão abordadas aplicações práticas nos diferentes níveis organizacionais. Por fim, apresentam-se as considerações finais e as referências utilizadas.